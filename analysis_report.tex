% Options for packages loaded elsewhere
\PassOptionsToPackage{unicode}{hyperref}
\PassOptionsToPackage{hyphens}{url}
\documentclass[
]{article}
\usepackage{xcolor}
\usepackage[margin=1in]{geometry}
\usepackage{amsmath,amssymb}
\setcounter{secnumdepth}{5}
\usepackage{iftex}
\ifPDFTeX
  \usepackage[T1]{fontenc}
  \usepackage[utf8]{inputenc}
  \usepackage{textcomp} % provide euro and other symbols
\else % if luatex or xetex
  \usepackage{unicode-math} % this also loads fontspec
  \defaultfontfeatures{Scale=MatchLowercase}
  \defaultfontfeatures[\rmfamily]{Ligatures=TeX,Scale=1}
\fi
\usepackage{lmodern}
\ifPDFTeX\else
  % xetex/luatex font selection
\fi
% Use upquote if available, for straight quotes in verbatim environments
\IfFileExists{upquote.sty}{\usepackage{upquote}}{}
\IfFileExists{microtype.sty}{% use microtype if available
  \usepackage[]{microtype}
  \UseMicrotypeSet[protrusion]{basicmath} % disable protrusion for tt fonts
}{}
\makeatletter
\@ifundefined{KOMAClassName}{% if non-KOMA class
  \IfFileExists{parskip.sty}{%
    \usepackage{parskip}
  }{% else
    \setlength{\parindent}{0pt}
    \setlength{\parskip}{6pt plus 2pt minus 1pt}}
}{% if KOMA class
  \KOMAoptions{parskip=half}}
\makeatother
\usepackage{color}
\usepackage{fancyvrb}
\newcommand{\VerbBar}{|}
\newcommand{\VERB}{\Verb[commandchars=\\\{\}]}
\DefineVerbatimEnvironment{Highlighting}{Verbatim}{commandchars=\\\{\}}
% Add ',fontsize=\small' for more characters per line
\usepackage{framed}
\definecolor{shadecolor}{RGB}{248,248,248}
\newenvironment{Shaded}{\begin{snugshade}}{\end{snugshade}}
\newcommand{\AlertTok}[1]{\textcolor[rgb]{0.94,0.16,0.16}{#1}}
\newcommand{\AnnotationTok}[1]{\textcolor[rgb]{0.56,0.35,0.01}{\textbf{\textit{#1}}}}
\newcommand{\AttributeTok}[1]{\textcolor[rgb]{0.13,0.29,0.53}{#1}}
\newcommand{\BaseNTok}[1]{\textcolor[rgb]{0.00,0.00,0.81}{#1}}
\newcommand{\BuiltInTok}[1]{#1}
\newcommand{\CharTok}[1]{\textcolor[rgb]{0.31,0.60,0.02}{#1}}
\newcommand{\CommentTok}[1]{\textcolor[rgb]{0.56,0.35,0.01}{\textit{#1}}}
\newcommand{\CommentVarTok}[1]{\textcolor[rgb]{0.56,0.35,0.01}{\textbf{\textit{#1}}}}
\newcommand{\ConstantTok}[1]{\textcolor[rgb]{0.56,0.35,0.01}{#1}}
\newcommand{\ControlFlowTok}[1]{\textcolor[rgb]{0.13,0.29,0.53}{\textbf{#1}}}
\newcommand{\DataTypeTok}[1]{\textcolor[rgb]{0.13,0.29,0.53}{#1}}
\newcommand{\DecValTok}[1]{\textcolor[rgb]{0.00,0.00,0.81}{#1}}
\newcommand{\DocumentationTok}[1]{\textcolor[rgb]{0.56,0.35,0.01}{\textbf{\textit{#1}}}}
\newcommand{\ErrorTok}[1]{\textcolor[rgb]{0.64,0.00,0.00}{\textbf{#1}}}
\newcommand{\ExtensionTok}[1]{#1}
\newcommand{\FloatTok}[1]{\textcolor[rgb]{0.00,0.00,0.81}{#1}}
\newcommand{\FunctionTok}[1]{\textcolor[rgb]{0.13,0.29,0.53}{\textbf{#1}}}
\newcommand{\ImportTok}[1]{#1}
\newcommand{\InformationTok}[1]{\textcolor[rgb]{0.56,0.35,0.01}{\textbf{\textit{#1}}}}
\newcommand{\KeywordTok}[1]{\textcolor[rgb]{0.13,0.29,0.53}{\textbf{#1}}}
\newcommand{\NormalTok}[1]{#1}
\newcommand{\OperatorTok}[1]{\textcolor[rgb]{0.81,0.36,0.00}{\textbf{#1}}}
\newcommand{\OtherTok}[1]{\textcolor[rgb]{0.56,0.35,0.01}{#1}}
\newcommand{\PreprocessorTok}[1]{\textcolor[rgb]{0.56,0.35,0.01}{\textit{#1}}}
\newcommand{\RegionMarkerTok}[1]{#1}
\newcommand{\SpecialCharTok}[1]{\textcolor[rgb]{0.81,0.36,0.00}{\textbf{#1}}}
\newcommand{\SpecialStringTok}[1]{\textcolor[rgb]{0.31,0.60,0.02}{#1}}
\newcommand{\StringTok}[1]{\textcolor[rgb]{0.31,0.60,0.02}{#1}}
\newcommand{\VariableTok}[1]{\textcolor[rgb]{0.00,0.00,0.00}{#1}}
\newcommand{\VerbatimStringTok}[1]{\textcolor[rgb]{0.31,0.60,0.02}{#1}}
\newcommand{\WarningTok}[1]{\textcolor[rgb]{0.56,0.35,0.01}{\textbf{\textit{#1}}}}
\usepackage{graphicx}
\makeatletter
\newsavebox\pandoc@box
\newcommand*\pandocbounded[1]{% scales image to fit in text height/width
  \sbox\pandoc@box{#1}%
  \Gscale@div\@tempa{\textheight}{\dimexpr\ht\pandoc@box+\dp\pandoc@box\relax}%
  \Gscale@div\@tempb{\linewidth}{\wd\pandoc@box}%
  \ifdim\@tempb\p@<\@tempa\p@\let\@tempa\@tempb\fi% select the smaller of both
  \ifdim\@tempa\p@<\p@\scalebox{\@tempa}{\usebox\pandoc@box}%
  \else\usebox{\pandoc@box}%
  \fi%
}
% Set default figure placement to htbp
\def\fps@figure{htbp}
\makeatother
\setlength{\emergencystretch}{3em} % prevent overfull lines
\providecommand{\tightlist}{%
  \setlength{\itemsep}{0pt}\setlength{\parskip}{0pt}}
\IfFileExists{bookmark.sty}{\usepackage{bookmark}}{}
\usepackage{bookmark}
\IfFileExists{xurl.sty}{\usepackage{xurl}}{} % add URL line breaks if available
\urlstyle{same}
\hypersetup{
  pdftitle={Relatório de Análise de Dados},
  pdfauthor={Lucas Cardoso - 179737},
  hidelinks,
  pdfcreator={LaTeX via pandoc}}

\title{Relatório de Análise de Dados}
\author{Lucas Cardoso - 179737}
\date{29 September, 2025}

\begin{document}
\maketitle

{
\setcounter{tocdepth}{2}
\tableofcontents
}
\section{Análise de Dados - Tidyng
Data}\label{anuxe1lise-de-dados---tidyng-data}

\subsection{\texorpdfstring{PARTE 1: Análise do Arquivo
\texttt{small\_file.txt}}{PARTE 1: Análise do Arquivo small\_file.txt}}\label{parte-1-anuxe1lise-do-arquivo-small_file.txt}

\subsubsection{Carregamento e Inspeção dos
Dados}\label{carregamento-e-inspeuxe7uxe3o-dos-dados}

Carregando o arquivo tab delimited:

\begin{Shaded}
\begin{Highlighting}[]
\NormalTok{my.data }\OtherTok{\textless{}{-}} \FunctionTok{read\_delim}\NormalTok{(file\_small, }\AttributeTok{delim =} \StringTok{"}\SpecialCharTok{\textbackslash{}t}\StringTok{"}\NormalTok{, }\AttributeTok{show\_col\_types =} \ConstantTok{FALSE}\NormalTok{)}

\FunctionTok{cat}\NormalTok{(}\StringTok{"Dados carregados com sucesso!}

\StringTok{"}\NormalTok{)}
\end{Highlighting}
\end{Shaded}

\begin{verbatim}
## Dados carregados com sucesso!
\end{verbatim}

Primeiras 6 linhas (\texttt{head()}):

\begin{Shaded}
\begin{Highlighting}[]
\FunctionTok{head}\NormalTok{(my.data)}
\end{Highlighting}
\end{Shaded}

\begin{verbatim}
## # A tibble: 6 x 3
##   Sample Length Category
##   <chr>   <dbl> <chr>   
## 1 x_1        45 A       
## 2 x_2        82 B       
## 3 x_3        81 C       
## 4 x_4        56 D       
## 5 x_5        96 A       
## 6 x_6        85 B
\end{verbatim}

Estrutura (\texttt{glimpse()}):

\begin{Shaded}
\begin{Highlighting}[]
\FunctionTok{glimpse}\NormalTok{(my.data)}
\end{Highlighting}
\end{Shaded}

\begin{verbatim}
## Rows: 40
## Columns: 3
## $ Sample   <chr> "x_1", "x_2", "x_3", "x_4", "x_5", "x_6", "x_7", "x_8", "x_9", "x_~
## $ Length   <dbl> 45, 82, 81, 56, 96, 85, 65, 96, 60, 62, 80, 63, 50, 64, 43, 98, 78~
## $ Category <chr> "A", "B", "C", "D", "A", "B", "C", "D", "A", "B", "C", "D", "A", "~
\end{verbatim}

\subsubsection{Filtros e Análises
Estatísticas}\label{filtros-e-anuxe1lises-estatuxedsticas}

Filtrando e ordenando dados:

Linhas da \textbf{Categoria D}:

\begin{Shaded}
\begin{Highlighting}[]
\NormalTok{category\_d }\OtherTok{\textless{}{-}}\NormalTok{ my.data }\SpecialCharTok{\%\textgreater{}\%} 
  \FunctionTok{filter}\NormalTok{(Category }\SpecialCharTok{==} \StringTok{"D"}\NormalTok{)}
\NormalTok{category\_d}
\end{Highlighting}
\end{Shaded}

\begin{verbatim}
## # A tibble: 10 x 3
##    Sample Length Category
##    <chr>   <dbl> <chr>   
##  1 x_4        56 D       
##  2 x_8        96 D       
##  3 x_12       63 D       
##  4 y_3        98 D       
##  5 y_7        79 D       
##  6 y_11       65 D       
##  7 z_2        83 D       
##  8 z_6        72 D       
##  9 z_10       84 D       
## 10 z_14       93 D
\end{verbatim}

Linhas da Categoria D ordenadas por Length:

\begin{Shaded}
\begin{Highlighting}[]
\NormalTok{category\_d\_ordered }\OtherTok{\textless{}{-}}\NormalTok{ my.data }\SpecialCharTok{\%\textgreater{}\%} 
  \FunctionTok{filter}\NormalTok{(Category }\SpecialCharTok{==} \StringTok{"D"}\NormalTok{) }\SpecialCharTok{\%\textgreater{}\%} 
  \FunctionTok{arrange}\NormalTok{(Length)}
\NormalTok{category\_d\_ordered}
\end{Highlighting}
\end{Shaded}

\begin{verbatim}
## # A tibble: 10 x 3
##    Sample Length Category
##    <chr>   <dbl> <chr>   
##  1 x_4        56 D       
##  2 x_12       63 D       
##  3 y_11       65 D       
##  4 z_6        72 D       
##  5 y_7        79 D       
##  6 z_2        83 D       
##  7 z_10       84 D       
##  8 z_14       93 D       
##  9 x_8        96 D       
## 10 y_3        98 D
\end{verbatim}

Cálculo das médias de Length:

\begin{Shaded}
\begin{Highlighting}[]
\NormalTok{mean\_length\_d }\OtherTok{\textless{}{-}}\NormalTok{ my.data }\SpecialCharTok{\%\textgreater{}\%} 
  \FunctionTok{filter}\NormalTok{(Category }\SpecialCharTok{==} \StringTok{"D"}\NormalTok{) }\SpecialCharTok{\%\textgreater{}\%} 
  \FunctionTok{pull}\NormalTok{(Length) }\SpecialCharTok{\%\textgreater{}\%} 
  \FunctionTok{mean}\NormalTok{()}

\NormalTok{mean\_length\_a }\OtherTok{\textless{}{-}}\NormalTok{ my.data }\SpecialCharTok{\%\textgreater{}\%} 
  \FunctionTok{filter}\NormalTok{(Category }\SpecialCharTok{==} \StringTok{"A"}\NormalTok{) }\SpecialCharTok{\%\textgreater{}\%} 
  \FunctionTok{pull}\NormalTok{(Length) }\SpecialCharTok{\%\textgreater{}\%} 
  \FunctionTok{mean}\NormalTok{()}

\FunctionTok{cat}\NormalTok{(}\StringTok{"Média do Length para categoria D:"}\NormalTok{, }\FunctionTok{round}\NormalTok{(mean\_length\_d, }\DecValTok{2}\NormalTok{), }\StringTok{"}\SpecialCharTok{\textbackslash{}n}\StringTok{"}\NormalTok{)}
\end{Highlighting}
\end{Shaded}

\begin{verbatim}
## Média do Length para categoria D: 78.9
\end{verbatim}

\begin{Shaded}
\begin{Highlighting}[]
\FunctionTok{cat}\NormalTok{(}\StringTok{"Média do Length para categoria A:"}\NormalTok{, }\FunctionTok{round}\NormalTok{(mean\_length\_a, }\DecValTok{2}\NormalTok{), }\StringTok{"}\SpecialCharTok{\textbackslash{}n}\StringTok{"}\NormalTok{)}
\end{Highlighting}
\end{Shaded}

\begin{verbatim}
## Média do Length para categoria A: 68.3
\end{verbatim}

\subsubsection{Visualizações
(small\_file.txt)}\label{visualizauxe7uxf5es-small_file.txt}

\paragraph{Distribuição de Length por Categoria
(Boxplot)}\label{distribuiuxe7uxe3o-de-length-por-categoria-boxplot}

\begin{Shaded}
\begin{Highlighting}[]
\NormalTok{p1 }\OtherTok{\textless{}{-}} \FunctionTok{ggplot}\NormalTok{(my.data, }\FunctionTok{aes}\NormalTok{(}\AttributeTok{x =}\NormalTok{ Category, }\AttributeTok{y =}\NormalTok{ Length, }\AttributeTok{fill =}\NormalTok{ Category)) }\SpecialCharTok{+}
  \FunctionTok{geom\_boxplot}\NormalTok{(}\AttributeTok{alpha =} \FloatTok{0.7}\NormalTok{) }\SpecialCharTok{+}
  \FunctionTok{geom\_jitter}\NormalTok{(}\AttributeTok{width =} \FloatTok{0.2}\NormalTok{, }\AttributeTok{alpha =} \FloatTok{0.6}\NormalTok{) }\SpecialCharTok{+}
  \FunctionTok{labs}\NormalTok{(}\AttributeTok{title =} \StringTok{"Distribuição de Length por Categoria"}\NormalTok{,}
       \AttributeTok{subtitle =} \StringTok{"small\_file.txt"}\NormalTok{,}
       \AttributeTok{x =} \StringTok{"Categoria"}\NormalTok{,}
       \AttributeTok{y =} \StringTok{"Length"}\NormalTok{) }\SpecialCharTok{+}
  \FunctionTok{theme\_minimal}\NormalTok{() }\SpecialCharTok{+}
  \FunctionTok{theme}\NormalTok{(}\AttributeTok{legend.position =} \StringTok{"none"}\NormalTok{)}

\FunctionTok{print}\NormalTok{(p1)}
\end{Highlighting}
\end{Shaded}

\pandocbounded{\includegraphics[keepaspectratio]{/Users/lucascardoso/Desktop/DATAVIZ/task 3/dataviz_tidy_data/analysis_report_files/figure-latex/parte1-visualization-1-1.pdf}}

\begin{Shaded}
\begin{Highlighting}[]
\CommentTok{\# ggsave("plots/length\_by\_category\_boxplot.png", p1, width = 8, height = 6, dpi = 300)}
\end{Highlighting}
\end{Shaded}

\paragraph{Média de Length por Categoria
(Barras)}\label{muxe9dia-de-length-por-categoria-barras}

\begin{Shaded}
\begin{Highlighting}[]
\NormalTok{means\_data }\OtherTok{\textless{}{-}}\NormalTok{ my.data }\SpecialCharTok{\%\textgreater{}\%}
  \FunctionTok{group\_by}\NormalTok{(Category) }\SpecialCharTok{\%\textgreater{}\%}
  \FunctionTok{summarise}\NormalTok{(}\AttributeTok{mean\_length =} \FunctionTok{mean}\NormalTok{(Length), }\AttributeTok{.groups =} \StringTok{"drop"}\NormalTok{)}

\NormalTok{p2 }\OtherTok{\textless{}{-}} \FunctionTok{ggplot}\NormalTok{(means\_data, }\FunctionTok{aes}\NormalTok{(}\AttributeTok{x =}\NormalTok{ Category, }\AttributeTok{y =}\NormalTok{ mean\_length, }\AttributeTok{fill =}\NormalTok{ Category)) }\SpecialCharTok{+}
  \FunctionTok{geom\_col}\NormalTok{(}\AttributeTok{alpha =} \FloatTok{0.8}\NormalTok{) }\SpecialCharTok{+}
  \FunctionTok{geom\_text}\NormalTok{(}\FunctionTok{aes}\NormalTok{(}\AttributeTok{label =} \FunctionTok{round}\NormalTok{(mean\_length, }\DecValTok{1}\NormalTok{)), }\AttributeTok{vjust =} \SpecialCharTok{{-}}\FloatTok{0.5}\NormalTok{) }\SpecialCharTok{+}
  \FunctionTok{labs}\NormalTok{(}\AttributeTok{title =} \StringTok{"Média de Length por Categoria"}\NormalTok{,}
       \AttributeTok{subtitle =} \StringTok{"small\_file.txt"}\NormalTok{,}
       \AttributeTok{x =} \StringTok{"Categoria"}\NormalTok{,}
       \AttributeTok{y =} \StringTok{"Média do Length"}\NormalTok{) }\SpecialCharTok{+}
  \FunctionTok{theme\_minimal}\NormalTok{() }\SpecialCharTok{+}
  \FunctionTok{theme}\NormalTok{(}\AttributeTok{legend.position =} \StringTok{"none"}\NormalTok{)}

\FunctionTok{print}\NormalTok{(p2)}
\end{Highlighting}
\end{Shaded}

\pandocbounded{\includegraphics[keepaspectratio]{/Users/lucascardoso/Desktop/DATAVIZ/task 3/dataviz_tidy_data/analysis_report_files/figure-latex/parte1-visualization-2-1.pdf}}

\begin{Shaded}
\begin{Highlighting}[]
\CommentTok{\# ggsave("plots/mean\_length\_by\_category.png", p2, width = 8, height = 6, dpi = 300)}
\end{Highlighting}
\end{Shaded}

\subsection{\texorpdfstring{PARTE 2: Análise e Tyding do Arquivo
\texttt{student\_grade.csv}}{PARTE 2: Análise e Tyding do Arquivo student\_grade.csv}}\label{parte-2-anuxe1lise-e-tyding-do-arquivo-student_grade.csv}

\subsubsection{Carregamento e
Inspeção}\label{carregamento-e-inspeuxe7uxe3o}

Carregando os dados de notas dos estudantes:

\begin{Shaded}
\begin{Highlighting}[]
\NormalTok{student\_data }\OtherTok{\textless{}{-}} \FunctionTok{read\_csv}\NormalTok{(file\_student, }\AttributeTok{show\_col\_types =} \ConstantTok{FALSE}\NormalTok{)}
\end{Highlighting}
\end{Shaded}

Primeiras 6 linhas dos dados originais:

\begin{Shaded}
\begin{Highlighting}[]
\FunctionTok{head}\NormalTok{(student\_data)}
\end{Highlighting}
\end{Shaded}

\begin{verbatim}
## # A tibble: 6 x 14
##    Year Class   Student      Q1    Q2    Q3    Q4    Q5    Q6    Q7    Q8    Q9   Q10
##   <dbl> <chr>   <chr>     <dbl> <dbl> <dbl> <dbl> <dbl> <dbl> <dbl> <dbl> <dbl> <dbl>
## 1  2022 Student Lucca      7.5   6.23  6.5   7.15 NA     5.43  8.58  8.19  7.96  7.92
## 2  2022 Student Salles    10    10    10    10    10    NA    10    10    10    10   
## 3  2022 Student Bueno      9.5   9     9     9.25  9.25  8     9.75  9.75  7.5   7.25
## 4  2022 Student Simas      9.5   9     9     9.25  9.25  8     9.75  9.75  7.5   7.25
## 5  2022 Student Goncalves  1.67  3.17  4.67  1.67  4     1.67  4.83  0.83  0.83  1.67
## 6  2022 Student Dornelles  9.1   8.75  9.83  9     9.75  9     9.5   9.25  9     9.18
## # i 1 more variable: Q11 <dbl>
\end{verbatim}

\subsubsection{Análise do Formato dos
Dados}\label{anuxe1lise-do-formato-dos-dados}

\begin{Shaded}
\begin{Highlighting}[]
\FunctionTok{cat}\NormalTok{(}\StringTok{"• Dimensões dos dados:"}\NormalTok{, }\FunctionTok{dim}\NormalTok{(student\_data), }\StringTok{"}\SpecialCharTok{\textbackslash{}n}\StringTok{"}\NormalTok{)}
\end{Highlighting}
\end{Shaded}

\begin{verbatim}
## • Dimensões dos dados: 43 14
\end{verbatim}

\begin{Shaded}
\begin{Highlighting}[]
\FunctionTok{cat}\NormalTok{(}\StringTok{"• Colunas:"}\NormalTok{, }\FunctionTok{colnames}\NormalTok{(student\_data), }\StringTok{"}\SpecialCharTok{\textbackslash{}n\textbackslash{}n}\StringTok{"}\NormalTok{)}
\end{Highlighting}
\end{Shaded}

\begin{verbatim}
## • Colunas: Year Class Student Q1 Q2 Q3 Q4 Q5 Q6 Q7 Q8 Q9 Q10 Q11
\end{verbatim}

\begin{Shaded}
\begin{Highlighting}[]
\FunctionTok{cat}\NormalTok{(}\StringTok{"• **Quais colunas são anotações e quais são medições?**}\SpecialCharTok{\textbackslash{}n}\StringTok{"}\NormalTok{)}
\end{Highlighting}
\end{Shaded}

\begin{verbatim}
## • **Quais colunas são anotações e quais são medições?**
\end{verbatim}

\begin{Shaded}
\begin{Highlighting}[]
\FunctionTok{cat}\NormalTok{(}\StringTok{"  {-} Anotações: Year, Class, Student (identificadores)}\SpecialCharTok{\textbackslash{}n}\StringTok{"}\NormalTok{)}
\end{Highlighting}
\end{Shaded}

\begin{verbatim}
##   - Anotações: Year, Class, Student (identificadores)
\end{verbatim}

\begin{Shaded}
\begin{Highlighting}[]
\FunctionTok{cat}\NormalTok{(}\StringTok{"  {-} Medições: Q1, Q2, Q3, Q4, Q5, Q6, Q7, Q8, Q9, Q10, Q11 (notas das questões)}\SpecialCharTok{\textbackslash{}n\textbackslash{}n}\StringTok{"}\NormalTok{)}
\end{Highlighting}
\end{Shaded}

\begin{verbatim}
##   - Medições: Q1, Q2, Q3, Q4, Q5, Q6, Q7, Q8, Q9, Q10, Q11 (notas das questões)
\end{verbatim}

\begin{Shaded}
\begin{Highlighting}[]
\FunctionTok{cat}\NormalTok{(}\StringTok{"• **Quantos tipos diferentes de medição existem?**}\SpecialCharTok{\textbackslash{}n}\StringTok{"}\NormalTok{)}
\end{Highlighting}
\end{Shaded}

\begin{verbatim}
## • **Quantos tipos diferentes de medição existem?**
\end{verbatim}

\begin{Shaded}
\begin{Highlighting}[]
\FunctionTok{cat}\NormalTok{(}\StringTok{"  {-} 1 tipo: Notas/pontuações das questões (todas são do mesmo tipo)}\SpecialCharTok{\textbackslash{}n\textbackslash{}n}\StringTok{"}\NormalTok{)}
\end{Highlighting}
\end{Shaded}

\begin{verbatim}
##   - 1 tipo: Notas/pontuações das questões (todas são do mesmo tipo)
\end{verbatim}

\begin{Shaded}
\begin{Highlighting}[]
\FunctionTok{cat}\NormalTok{(}\StringTok{"• **Todas as medições do mesmo tipo estão em uma única coluna?**}\SpecialCharTok{\textbackslash{}n}\StringTok{"}\NormalTok{)}
\end{Highlighting}
\end{Shaded}

\begin{verbatim}
## • **Todas as medições do mesmo tipo estão em uma única coluna?**
\end{verbatim}

\begin{Shaded}
\begin{Highlighting}[]
\FunctionTok{cat}\NormalTok{(}\StringTok{"  {-} NÃO. As notas estão espalhadas em 11 colunas (Q1 a Q11). Os dados estão em formato *wide*.}\SpecialCharTok{\textbackslash{}n\textbackslash{}n}\StringTok{"}\NormalTok{)}
\end{Highlighting}
\end{Shaded}

\begin{verbatim}
##   - NÃO. As notas estão espalhadas em 11 colunas (Q1 a Q11). Os dados estão em formato *wide*.
\end{verbatim}

\begin{Shaded}
\begin{Highlighting}[]
\FunctionTok{cat}\NormalTok{(}\StringTok{"• **Qual é o nome da variável sendo medida?**}\SpecialCharTok{\textbackslash{}n}\StringTok{"}\NormalTok{)}
\end{Highlighting}
\end{Shaded}

\begin{verbatim}
## • **Qual é o nome da variável sendo medida?**
\end{verbatim}

\begin{Shaded}
\begin{Highlighting}[]
\FunctionTok{cat}\NormalTok{(}\StringTok{"  {-} \textquotesingle{}Grade\textquotesingle{} ou \textquotesingle{}Nota\textquotesingle{} (pontuação obtida pelo estudante)}\SpecialCharTok{\textbackslash{}n\textbackslash{}n}\StringTok{"}\NormalTok{)}
\end{Highlighting}
\end{Shaded}

\begin{verbatim}
##   - 'Grade' ou 'Nota' (pontuação obtida pelo estudante)
\end{verbatim}

\begin{Shaded}
\begin{Highlighting}[]
\FunctionTok{cat}\NormalTok{(}\StringTok{"• **O nome da variável está em uma coluna?**}\SpecialCharTok{\textbackslash{}n}\StringTok{"}\NormalTok{)}
\end{Highlighting}
\end{Shaded}

\begin{verbatim}
## • **O nome da variável está em uma coluna?**
\end{verbatim}

\begin{Shaded}
\begin{Highlighting}[]
\FunctionTok{cat}\NormalTok{(}\StringTok{"  {-} NÃO. Os nomes estão nos cabeçalhos das colunas (Q1, Q2, etc.)}\SpecialCharTok{\textbackslash{}n\textbackslash{}n}\StringTok{"}\NormalTok{)}
\end{Highlighting}
\end{Shaded}

\begin{verbatim}
##   - NÃO. Os nomes estão nos cabeçalhos das colunas (Q1, Q2, etc.)
\end{verbatim}

\subsubsection{Transformação para Formato
Tidy}\label{transformauxe7uxe3o-para-formato-tidy}

\begin{Shaded}
\begin{Highlighting}[]
\NormalTok{student\_tidy }\OtherTok{\textless{}{-}}\NormalTok{ student\_data }\SpecialCharTok{\%\textgreater{}\%}
  \CommentTok{\# Transformar de wide para long format}
  \FunctionTok{pivot\_longer}\NormalTok{(}\AttributeTok{cols =}\NormalTok{ Q1}\SpecialCharTok{:}\NormalTok{Q11, }
               \AttributeTok{names\_to =} \StringTok{"Question"}\NormalTok{, }
               \AttributeTok{values\_to =} \StringTok{"Grade"}\NormalTok{) }\SpecialCharTok{\%\textgreater{}\%}
  \CommentTok{\# Remover linhas com valores NA}
  \FunctionTok{filter}\NormalTok{(}\SpecialCharTok{!}\FunctionTok{is.na}\NormalTok{(Grade)) }\SpecialCharTok{\%\textgreater{}\%}
  \CommentTok{\# Remover colunas com informação repetida (se houver)}
  \FunctionTok{select}\NormalTok{(Year, Class, Student, Question, Grade)}

\FunctionTok{cat}\NormalTok{(}\StringTok{"Dados em formato tidy (primeiras 10 linhas):}\SpecialCharTok{\textbackslash{}n}\StringTok{"}\NormalTok{)}
\end{Highlighting}
\end{Shaded}

\begin{verbatim}
## Dados em formato tidy (primeiras 10 linhas):
\end{verbatim}

\begin{Shaded}
\begin{Highlighting}[]
\FunctionTok{head}\NormalTok{(student\_tidy, }\DecValTok{10}\NormalTok{)}
\end{Highlighting}
\end{Shaded}

\begin{verbatim}
## # A tibble: 10 x 5
##     Year Class   Student Question Grade
##    <dbl> <chr>   <chr>   <chr>    <dbl>
##  1  2022 Student Lucca   Q1        7.5 
##  2  2022 Student Lucca   Q2        6.23
##  3  2022 Student Lucca   Q3        6.5 
##  4  2022 Student Lucca   Q4        7.15
##  5  2022 Student Lucca   Q6        5.43
##  6  2022 Student Lucca   Q7        8.58
##  7  2022 Student Lucca   Q8        8.19
##  8  2022 Student Lucca   Q9        7.96
##  9  2022 Student Lucca   Q10       7.92
## 10  2022 Student Lucca   Q11       6.48
\end{verbatim}

\subsubsection{Estatísticas e
Performance}\label{estatuxedsticas-e-performance}

Estatísticas das Questões 1 e 2:

\begin{Shaded}
\begin{Highlighting}[]
\NormalTok{q1\_q2\_stats }\OtherTok{\textless{}{-}}\NormalTok{ student\_tidy }\SpecialCharTok{\%\textgreater{}\%}
  \FunctionTok{filter}\NormalTok{(Question }\SpecialCharTok{\%in\%} \FunctionTok{c}\NormalTok{(}\StringTok{"Q1"}\NormalTok{, }\StringTok{"Q2"}\NormalTok{)) }\SpecialCharTok{\%\textgreater{}\%}
  \FunctionTok{group\_by}\NormalTok{(Question) }\SpecialCharTok{\%\textgreater{}\%}
  \FunctionTok{summarise}\NormalTok{(}
    \AttributeTok{mean\_grade =} \FunctionTok{mean}\NormalTok{(Grade, }\AttributeTok{na.rm =} \ConstantTok{TRUE}\NormalTok{),}
    \AttributeTok{sd\_grade =} \FunctionTok{sd}\NormalTok{(Grade, }\AttributeTok{na.rm =} \ConstantTok{TRUE}\NormalTok{),}
    \AttributeTok{n\_students =} \FunctionTok{n}\NormalTok{(),}
    \AttributeTok{.groups =} \StringTok{"drop"}
\NormalTok{  )}

\FunctionTok{cat}\NormalTok{(}\StringTok{"Estatísticas Q1 e Q2:}\SpecialCharTok{\textbackslash{}n}\StringTok{"}\NormalTok{)}
\end{Highlighting}
\end{Shaded}

\begin{verbatim}
## Estatísticas Q1 e Q2:
\end{verbatim}

\begin{Shaded}
\begin{Highlighting}[]
\FunctionTok{print}\NormalTok{(q1\_q2\_stats)}
\end{Highlighting}
\end{Shaded}

\begin{verbatim}
## # A tibble: 2 x 4
##   Question mean_grade sd_grade n_students
##   <chr>         <dbl>    <dbl>      <int>
## 1 Q1             8.50     1.61         43
## 2 Q2             7.95     1.62         43
\end{verbatim}

Dificuldade das Questões (ordenado da mais difícil para a mais fácil
pela média):

\begin{Shaded}
\begin{Highlighting}[]
\NormalTok{question\_difficulty }\OtherTok{\textless{}{-}}\NormalTok{ student\_tidy }\SpecialCharTok{\%\textgreater{}\%}
  \FunctionTok{group\_by}\NormalTok{(Question) }\SpecialCharTok{\%\textgreater{}\%}
  \FunctionTok{summarise}\NormalTok{(}
    \AttributeTok{mean\_grade =} \FunctionTok{mean}\NormalTok{(Grade),}
    \AttributeTok{sd\_grade =} \FunctionTok{sd}\NormalTok{(Grade),}
    \AttributeTok{n\_students =} \FunctionTok{n}\NormalTok{(),}
    \AttributeTok{.groups =} \StringTok{"drop"}
\NormalTok{  ) }\SpecialCharTok{\%\textgreater{}\%}
  \FunctionTok{arrange}\NormalTok{(mean\_grade)}

\FunctionTok{print}\NormalTok{(question\_difficulty)}
\end{Highlighting}
\end{Shaded}

\begin{verbatim}
## # A tibble: 11 x 4
##    Question mean_grade sd_grade n_students
##    <chr>         <dbl>    <dbl>      <int>
##  1 Q6             7.32     1.93         41
##  2 Q2             7.95     1.62         43
##  3 Q11            8.02     1.72         43
##  4 Q4             8.18     1.63         43
##  5 Q10            8.18     1.61         43
##  6 Q5             8.30     1.48         42
##  7 Q9             8.37     1.67         43
##  8 Q8             8.39     1.73         43
##  9 Q1             8.50     1.61         43
## 10 Q3             8.57     1.38         42
## 11 Q7             9.04     1.16         43
\end{verbatim}

Performance geral por estudante:

\begin{Shaded}
\begin{Highlighting}[]
\NormalTok{student\_performance }\OtherTok{\textless{}{-}}\NormalTok{ student\_tidy }\SpecialCharTok{\%\textgreater{}\%}
  \FunctionTok{group\_by}\NormalTok{(Student) }\SpecialCharTok{\%\textgreater{}\%}
  \FunctionTok{summarise}\NormalTok{(}
    \AttributeTok{mean\_grade =} \FunctionTok{mean}\NormalTok{(Grade),}
    \AttributeTok{n\_questions =} \FunctionTok{n}\NormalTok{(),}
    \AttributeTok{.groups =} \StringTok{"drop"}
\NormalTok{  ) }\SpecialCharTok{\%\textgreater{}\%}
  \FunctionTok{arrange}\NormalTok{(}\FunctionTok{desc}\NormalTok{(mean\_grade))}

\FunctionTok{cat}\NormalTok{(}\StringTok{"Top 5 estudantes por média geral:}\SpecialCharTok{\textbackslash{}n}\StringTok{"}\NormalTok{)}
\end{Highlighting}
\end{Shaded}

\begin{verbatim}
## Top 5 estudantes por média geral:
\end{verbatim}

\begin{Shaded}
\begin{Highlighting}[]
\FunctionTok{head}\NormalTok{(student\_performance, }\DecValTok{5}\NormalTok{)}
\end{Highlighting}
\end{Shaded}

\begin{verbatim}
## # A tibble: 5 x 3
##   Student   mean_grade n_questions
##   <chr>          <dbl>       <int>
## 1 Junior         10             10
## 2 Salles         10             10
## 3 Pedro           9.81          11
## 4 Gabriel         9.70          11
## 5 Francisca       9.64          11
\end{verbatim}

\begin{Shaded}
\begin{Highlighting}[]
\FunctionTok{cat}\NormalTok{(}\StringTok{"}\SpecialCharTok{\textbackslash{}n}\StringTok{Bottom 5 estudantes por média geral:}\SpecialCharTok{\textbackslash{}n}\StringTok{"}\NormalTok{)}
\end{Highlighting}
\end{Shaded}

\begin{verbatim}
## 
## Bottom 5 estudantes por média geral:
\end{verbatim}

\begin{Shaded}
\begin{Highlighting}[]
\FunctionTok{tail}\NormalTok{(student\_performance, }\DecValTok{5}\NormalTok{)}
\end{Highlighting}
\end{Shaded}

\begin{verbatim}
## # A tibble: 5 x 3
##   Student   mean_grade n_questions
##   <chr>          <dbl>       <int>
## 1 Rafaela         6.92          11
## 2 Ramos           6.13          11
## 3 Samara          5.48          11
## 4 Gleiser         4.42          11
## 5 Goncalves       2.43          11
\end{verbatim}

\subsubsection{Visualizações
(student\_grade.csv)}\label{visualizauxe7uxf5es-student_grade.csv}

\paragraph{Distribuição das Notas por Questão
(Boxplot)}\label{distribuiuxe7uxe3o-das-notas-por-questuxe3o-boxplot}

\begin{Shaded}
\begin{Highlighting}[]
\NormalTok{p3 }\OtherTok{\textless{}{-}} \FunctionTok{ggplot}\NormalTok{(student\_tidy, }\FunctionTok{aes}\NormalTok{(}\AttributeTok{x =}\NormalTok{ Question, }\AttributeTok{y =}\NormalTok{ Grade, }\AttributeTok{fill =}\NormalTok{ Question)) }\SpecialCharTok{+}
  \FunctionTok{geom\_boxplot}\NormalTok{(}\AttributeTok{alpha =} \FloatTok{0.7}\NormalTok{) }\SpecialCharTok{+}
  \FunctionTok{labs}\NormalTok{(}\AttributeTok{title =} \StringTok{"Distribuição das Notas por Questão"}\NormalTok{,}
       \AttributeTok{subtitle =} \StringTok{"student\_grade.csv {-} Dados Tidy"}\NormalTok{,}
       \AttributeTok{x =} \StringTok{"Questão"}\NormalTok{,}
       \AttributeTok{y =} \StringTok{"Nota"}\NormalTok{) }\SpecialCharTok{+}
  \FunctionTok{theme\_minimal}\NormalTok{() }\SpecialCharTok{+}
  \FunctionTok{theme}\NormalTok{(}\AttributeTok{axis.text.x =} \FunctionTok{element\_text}\NormalTok{(}\AttributeTok{angle =} \DecValTok{45}\NormalTok{, }\AttributeTok{hjust =} \DecValTok{1}\NormalTok{),}
        \AttributeTok{legend.position =} \StringTok{"none"}\NormalTok{) }\SpecialCharTok{+}
  \FunctionTok{scale\_y\_continuous}\NormalTok{(}\AttributeTok{limits =} \FunctionTok{c}\NormalTok{(}\DecValTok{0}\NormalTok{, }\DecValTok{10}\NormalTok{))}

\FunctionTok{print}\NormalTok{(p3)}
\end{Highlighting}
\end{Shaded}

\pandocbounded{\includegraphics[keepaspectratio]{/Users/lucascardoso/Desktop/DATAVIZ/task 3/dataviz_tidy_data/analysis_report_files/figure-latex/parte2-visualization-1-1.pdf}}

\paragraph{Dificuldade das Questões
(Barras)}\label{dificuldade-das-questuxf5es-barras}

\begin{Shaded}
\begin{Highlighting}[]
\NormalTok{p5 }\OtherTok{\textless{}{-}} \FunctionTok{ggplot}\NormalTok{(question\_difficulty, }\FunctionTok{aes}\NormalTok{(}\AttributeTok{x =} \FunctionTok{reorder}\NormalTok{(Question, mean\_grade), }\AttributeTok{y =}\NormalTok{ mean\_grade)) }\SpecialCharTok{+}
  \FunctionTok{geom\_col}\NormalTok{(}\AttributeTok{fill =} \StringTok{"steelblue"}\NormalTok{, }\AttributeTok{alpha =} \FloatTok{0.8}\NormalTok{) }\SpecialCharTok{+}
  \FunctionTok{geom\_text}\NormalTok{(}\FunctionTok{aes}\NormalTok{(}\AttributeTok{label =} \FunctionTok{round}\NormalTok{(mean\_grade, }\DecValTok{2}\NormalTok{)), }\AttributeTok{hjust =} \SpecialCharTok{{-}}\FloatTok{0.1}\NormalTok{) }\SpecialCharTok{+}
  \FunctionTok{coord\_flip}\NormalTok{() }\SpecialCharTok{+}
  \FunctionTok{labs}\NormalTok{(}\AttributeTok{title =} \StringTok{"Dificuldade das Questões"}\NormalTok{,}
       \AttributeTok{subtitle =} \StringTok{"Média das notas por questão (ordenado da mais difícil para a mais fácil)"}\NormalTok{,}
       \AttributeTok{x =} \StringTok{"Questão"}\NormalTok{,}
       \AttributeTok{y =} \StringTok{"Média das Notas"}\NormalTok{) }\SpecialCharTok{+}
  \FunctionTok{theme\_minimal}\NormalTok{()}

\FunctionTok{print}\NormalTok{(p5)}
\end{Highlighting}
\end{Shaded}

\pandocbounded{\includegraphics[keepaspectratio]{/Users/lucascardoso/Desktop/DATAVIZ/task 3/dataviz_tidy_data/analysis_report_files/figure-latex/parte2-visualization-2-1.pdf}}

\paragraph{Heatmap de Performance dos
Estudantes}\label{heatmap-de-performance-dos-estudantes}

\begin{Shaded}
\begin{Highlighting}[]
\NormalTok{grade\_heatmap\_data }\OtherTok{\textless{}{-}}\NormalTok{ student\_tidy }\SpecialCharTok{\%\textgreater{}\%}
  \FunctionTok{arrange}\NormalTok{(Student, Question)}

\NormalTok{p6 }\OtherTok{\textless{}{-}} \FunctionTok{ggplot}\NormalTok{(grade\_heatmap\_data, }\FunctionTok{aes}\NormalTok{(}\AttributeTok{x =}\NormalTok{ Question, }\AttributeTok{y =}\NormalTok{ Student, }\AttributeTok{fill =}\NormalTok{ Grade)) }\SpecialCharTok{+}
  \FunctionTok{geom\_tile}\NormalTok{(}\AttributeTok{color =} \StringTok{"white"}\NormalTok{) }\SpecialCharTok{+}
  \FunctionTok{scale\_fill\_gradient2}\NormalTok{(}\AttributeTok{low =} \StringTok{"red"}\NormalTok{, }\AttributeTok{mid =} \StringTok{"yellow"}\NormalTok{, }\AttributeTok{high =} \StringTok{"green"}\NormalTok{, }
                       \AttributeTok{midpoint =} \DecValTok{5}\NormalTok{, }\AttributeTok{name =} \StringTok{"Nota"}\NormalTok{) }\SpecialCharTok{+}
  \FunctionTok{labs}\NormalTok{(}\AttributeTok{title =} \StringTok{"Heatmap de Performance dos Estudantes"}\NormalTok{,}
       \AttributeTok{subtitle =} \StringTok{"Notas por estudante e questão"}\NormalTok{,}
       \AttributeTok{x =} \StringTok{"Questão"}\NormalTok{,}
       \AttributeTok{y =} \StringTok{"Estudante"}\NormalTok{) }\SpecialCharTok{+}
  \FunctionTok{theme\_minimal}\NormalTok{() }\SpecialCharTok{+}
  \FunctionTok{theme}\NormalTok{(}\AttributeTok{axis.text.y =} \FunctionTok{element\_text}\NormalTok{(}\AttributeTok{size =} \DecValTok{6}\NormalTok{),}
        \AttributeTok{axis.text.x =} \FunctionTok{element\_text}\NormalTok{(}\AttributeTok{angle =} \DecValTok{45}\NormalTok{, }\AttributeTok{hjust =} \DecValTok{1}\NormalTok{))}

\FunctionTok{print}\NormalTok{(p6)}
\end{Highlighting}
\end{Shaded}

\pandocbounded{\includegraphics[keepaspectratio]{/Users/lucascardoso/Desktop/DATAVIZ/task 3/dataviz_tidy_data/analysis_report_files/figure-latex/parte2-visualization-3-1.pdf}}

\paragraph{Comparação Q1 vs Q2}\label{comparauxe7uxe3o-q1-vs-q2}

\begin{Shaded}
\begin{Highlighting}[]
\NormalTok{q1\_q2\_data }\OtherTok{\textless{}{-}}\NormalTok{ student\_tidy }\SpecialCharTok{\%\textgreater{}\%}
  \FunctionTok{filter}\NormalTok{(Question }\SpecialCharTok{\%in\%} \FunctionTok{c}\NormalTok{(}\StringTok{"Q1"}\NormalTok{, }\StringTok{"Q2"}\NormalTok{))}

\NormalTok{p7 }\OtherTok{\textless{}{-}} \FunctionTok{ggplot}\NormalTok{(q1\_q2\_data, }\FunctionTok{aes}\NormalTok{(}\AttributeTok{x =}\NormalTok{ Question, }\AttributeTok{y =}\NormalTok{ Grade, }\AttributeTok{fill =}\NormalTok{ Question)) }\SpecialCharTok{+}
  \FunctionTok{geom\_violin}\NormalTok{(}\AttributeTok{alpha =} \FloatTok{0.7}\NormalTok{) }\SpecialCharTok{+}
  \FunctionTok{geom\_boxplot}\NormalTok{(}\AttributeTok{width =} \FloatTok{0.2}\NormalTok{, }\AttributeTok{alpha =} \FloatTok{0.9}\NormalTok{) }\SpecialCharTok{+}
  \FunctionTok{stat\_summary}\NormalTok{(}\AttributeTok{fun =}\NormalTok{ mean, }\AttributeTok{geom =} \StringTok{"point"}\NormalTok{, }\AttributeTok{shape =} \DecValTok{23}\NormalTok{, }\AttributeTok{size =} \DecValTok{3}\NormalTok{, }\AttributeTok{fill =} \StringTok{"white"}\NormalTok{) }\SpecialCharTok{+}
  \FunctionTok{labs}\NormalTok{(}\AttributeTok{title =} \StringTok{"Comparação entre Q1 e Q2"}\NormalTok{,}
       \AttributeTok{subtitle =} \StringTok{"Distribuição das notas com médias destacadas"}\NormalTok{,}
       \AttributeTok{x =} \StringTok{"Questão"}\NormalTok{,}
       \AttributeTok{y =} \StringTok{"Nota"}\NormalTok{) }\SpecialCharTok{+}
  \FunctionTok{theme\_minimal}\NormalTok{() }\SpecialCharTok{+}
  \FunctionTok{theme}\NormalTok{(}\AttributeTok{legend.position =} \StringTok{"none"}\NormalTok{)}

\FunctionTok{print}\NormalTok{(p7)}
\end{Highlighting}
\end{Shaded}

\pandocbounded{\includegraphics[keepaspectratio]{/Users/lucascardoso/Desktop/DATAVIZ/task 3/dataviz_tidy_data/analysis_report_files/figure-latex/parte2-visualization-4-1.pdf}}

FIM DO RELATÓRIO

\end{document}
